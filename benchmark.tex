\section{Reporting Benchmark Results}
\label{sec:benchmark}

This section provides instructions for how to preprare and report benchmark
results.

When comparing against previously published benchmarks, it is best to to use the
same version of \texttt{cleverhans} as was used to produce the previous
benchmarks. This minimizes the possibility that an undetected change in behavior
between versions could cause a difference in the output of the benchmark
results.

When reporting new results that are not directly compared to previous work, it
is best to use the most recent versioned release of \texttt{cleverhans}.

In all cases, it is important to report the version number of
\texttt{cleverhans}.

In addition to this information, one should also report which attack methods
were used, and the values of any configuration parameters used for these
attacks.

For example, you might report ``We benchmarked the robustness of our method to
adversarial attack using v1.0.0 of \texttt{cleverhans} (Goodfellow et al. 2016).
On a test set modified by \texttt{fgsm} with \texttt{eps} of 0.3, we obtained a
test set accuracy of 71.3\%.''

The library does not provide specific test datasets or data preprocessing. End
users are responsible for appropriately preparing the data in their specific
application areas, and for reporting sufficient information about the data
preprocessing and model family to make benchmarks appropriately comparable.
